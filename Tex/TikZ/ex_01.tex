%!TEX encoding = UTF-8 Unicode
\documentclass[ % options,
    a4paper,    % papersize
%    cjk,       % for cjk-ko
%    usedotemph,% for cjk-ko's \dotemph
    amsmath,    % load amsmath.sty to typeset math materials
    itemph,     % to disable gremph default (xe/lua)
%    footnote,  % korean style footnote
%    chapter,   % to use \chapter
]{oblivoir}     % xoblivoir and oblivoir are identical.

\ifPDFTeX       % latex, pdflatex
%    \usepackage{newtxtext}    % Latin fonts
\else\ifLuaOrXeTeX   % xelatex or lualatex
%  \setmainfont{TeX Gyre Termes}   %% Latin fonts
%	\setkomainfont(Noto Serif CJK KR)(* Bold)(* Medium)
%	\setkosansfont(Noto Sans CJK KR)(* Bold)(* Medium)
\fi\fi

% packages
\usepackage{kotex-logo}
\usepackage[utf]{kotex}
\usepackage{geometry}
 \geometry{
 a4paper,
 total={170mm,257mm},
 left=20mm,
 top=10mm,
 }
 
%% font packages and setup
\usepackage{fontspec}
\setmainfont{UnDotum}
\setsansfont{UnDotum}
\setmonofont{UnTaza}
\usepackage{dhucs-interword}
\interhword[.6]{.475}{.1}{.1}
\setlength{\parindent}{0em}
\setlength{\parskip}{1em}

% operator
\DeclareMathOperator*{\argmax}{argmax}
\DeclareMathOperator{\E}{\mathbb{E}}

% tikz
\usepackage{tikz}
\usetikzlibrary{mindmap}

\begin{document}

\title{Graphics 연습}
\author{wisemountain}
\date{\today}

\maketitle

\newpage

\tableofcontents

\newpage

\section{기본기 다지기}

Tex 문서에 깔끔하게 만든 그림이 포함되면 개념을 깔끔하게 이미지로 
전달할 수 있기 때문에 문서의 품질이 올라간다. 적절한 이미지의 일관된 
사용은 문서의 시인성도 올리는 효과를 갖는다. 그래서 전체적으로 읽기 
편한 문서로 만들 수 있다. 

Tikz와 pgf는 latex에서 그리기를 편리하게 해주는 다양한 라이브러리 모음이고 
일관된 개념하에 개발되었다. 

\subsection{튜토리얼 1}

\begin{tikzpicture}
\draw (0, 0) -- (4, 0)  -- (4, 4) -- (0, 4) -- (0, 0);
\end{tikzpicture}

\begin{tikzpicture}
\draw (2, 5) circle(3cm);
\end{tikzpicture}

\begin{tikzpicture}
\draw[step=1cm, gray, very thin] (-2, -2) grid(6, 6);
\fill[blue!40!white] (0, 0) rectangle(4, 4);
\end{tikzpicture}


\subsection{마인드맵} 

간단한 라이브러리이지만 중간 중간 개념 정리나 방향을 보여주기에 
좋은 라이브러리이다. 양념으로 매우 좋아 보인다. 

\begin{tikzpicture}[grow cyclic, every node/.style=concept, concept color=blue!40, 
	level 1/.style={level distance=5cm, sibling angle=30}, 
	level 2/.style={level distance=3cm, sibling angle=35}]

\node {ShareLaTeX Tutorial Videos}
	child { node { Beginners Series } }
	child { node { Thesis Series } }
	child { node { TikZ Series } 
		child { node { Test 1 } }
	}
	child { node { Beamer Series } }	
	child { node { Diff Forms Series } }	
;
	
\end{tikzpicture}


\section{튜토리얼 2}

https://www.math.uni-leipzig.de/~hellmund/LaTeX/pgf-tut.pdf 여기에 있는 내용이 
핵심 개념 위주로 매우 잘 설명되어 있다. Tikz와 pgf로 불가능한 것은 없어 보인다. 

path, node 두 가지가 핵심 개념이고 이걸로 모든 것이 가능하다. 참 깔끔하고 
유용한 라이브러리이다. 



\end{document}
